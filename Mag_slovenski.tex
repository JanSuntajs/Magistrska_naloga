%%%%%%%%%%%%%%%%%%%%%%%%%%%%%%%%%%%%%%%%%%%%%%%%%%%%%%%%%%          METAPODATKI

\begin{filecontents*}{\jobname.xmpdata}
\Title{Naslov magistrskega dela}
\Author{Ime in priimek}
\Keywords{magisterij\sep navodila\sep ključne besede}
\Publisher{Univerza v Ljubljani, Fakulteta za matematiko in fiziko}
\end{filecontents*}

%%%%%%%%%%%%%%%%%%%%%%%%%%%%%%%%%%%%%%%%%%%%%%%%%%%%%%%%%%

\documentclass[longbibliography,slovene,a4paper,12pt]{book}
%\usepackage[english]{babel}    % angleski delilni vzorci
\usepackage[slovene]{babel}    % slovenski delilni vzorci
\usepackage[utf8]{inputenc}
\usepackage{amsfonts}
\usepackage[T1]{fontenc}
\usepackage[pdftex]{graphicx}
\usepackage{fancyhdr}
\usepackage[sort, numbers]{natbib}
\include{template}

%%%%%%%%%%%%%%%%%%%%%%%%%%%%%%%%%%%%%%%%%%%%%%%%%%%%%%%%%%       PDF/A

\usepackage{xmpincl}
\usepackage[a-1b]{pdfx}       

%%%%%%%%%%%%%%%%%%%%%%%%%%%%%%%%%%%%%%%%%%%%%%%%%%%%%%%%%%

\usepackage{hyperref}
\usepackage[a4paper,inner=3.5cm,outer=2.5cm,top=2.5cm,bottom=2.5cm,pdftex]{geometry}
\usepackage[titletoc,title]{appendix}
\usepackage{epstopdf}
\usepackage{url}
\usepackage{makeidx}
\pagestyle{headings}
\makeindex

%%
%% Za pisanje sumnikov imamo tri moznosti:
%%   --- vnasamo jih neposredno v kodnem sistemu UTF-8 
%%   --- pisemo jih z latexovim ukazom, ki je namenjen natanko temu,
%%       in sicer kot \v{c}, \v{s}, \v{z}, \v{C}, \v{S}, v{\Z} ali
%%       malo manj pregledno kot \v c, \v s, \v z, \v C, \v S, \v Z,
%%   --- pisemo jih kot "c, "s, "z, "C, "S, "Z), vendar tedaj potrebujemo
%%       spodaj zapisani macro, ki znaku " pripise vlogo `izdelave' sumnika:
% \catcode`\"=\active\def"#1{\v{#1}}
%%       torej \v{S}krjan\v{c}ek == \v Skrjan\v cek == "Skrjan"cek
%% Pozor: narekovaj potem ne smemo vec pisati kot " ampak kot `` in '',
%%       torej: "Skrjan"cek je "civkal ``"ci-"ci-"ci''.

%%
%% Mozni nacini stevilcenja strani:
%%  --- arabske stevilke v celotnem dokumentu, kot je uporabljeno v tej predlogi
%%  --- del strani je lahko stevilcen z rimskimi stevilkami, razen uvoda, osrednjega dela, zakljucka in seznama literature.
%% V vsakem primeru se stevilke na straneh izpisejo sele od kazala naprej.

\def\epsfg#1#2{\epsfig{file=#1.eps,width=#2}}
\def\legendamp#1#2{\vbox{\hsize=#1\caption{\small #2}}}

\setcounter{topnumber}{4}
\setcounter{bottomnumber}{4}
\setcounter{totalnumber}{5}
\renewcommand{\topfraction}{0.99}
\renewcommand{\bottomfraction}{0.99}
\renewcommand{\textfraction}{0.0}
\setlength{\tabcolsep}{10pt}
\renewcommand{\arraystretch}{1.5}

\def\bi#1{\hbox{\boldmath{$#1$}}}
\let\oldvec\vec
\def\vec#1{\mbox{\boldmath$#1$}}
\def\pol{{\textstyle{1\over2}}}
\def\svec#1{\mbox{{\scriptsize \boldmath$#1$}}}

\begin{document}

%%% NASLOVNA STRAN

\pagestyle{empty}
\begin{center}

{\large UNIVERZA V LJUBLJANI\\
FAKULTETA ZA MATEMATIKO IN FIZIKO\\
ODDELEK ZA FIZIKO\\
PROGRAM in SMER ŠTUDIJA\\}


\vspace{4cm}


{\Large Ime in priimek\\}

\vspace{10mm}

{\bf \Large NASLOV MAGISTRSKEGA DELA}\\
\vspace{5mm}
{\large Magistrsko delo}\\




\vfill



{\large MENTOR$\backslash$-ICA: naziv, Ime in priimek\\
SOMENTOR$\backslash$-ICA: naziv, Ime in priimek\\


\vspace{2cm}
Ljubljana, leto}

\end{center}

%%% ZAHVALA (NEOBVEZNO)

\cleardoublepage
\mbox{}
\vfill
{\Large \bf Zahvala}
\vspace{1cm}\\

%%% IZVLECEK

\cleardoublepage
{\Large \bf Izvleček}
\vspace{1cm}\\
Kratek izvleček v slovenskem jeziku.\\
\vspace{1cm}\\
{\bf Ključne besede:}\\
{\bf PACS:}

%%% ABSTRACT

\cleardoublepage
{\Large \bf Abstract}
\vspace{1cm}\\
Kratek izvleček v angleškem jeziku.
\vspace{1cm}\\
{\bf Keywords:}\\
{\bf PACS:}

%%% KAZALO

\tableofcontents

%%% SEZNAM SLIK (NEOBVEZNO)

\cleardoublepage\phantomsection
\renewcommand\listfigurename{Seznam slik}
\addcontentsline{toc}{chapter}{\listfigurename}
\listoffigures

%%% SEZNAM TABEL (NEOBVEZNO)

\cleardoublepage\phantomsection
\renewcommand\listtablename{Seznam tabel}
\addcontentsline{toc}{chapter}{\listtablename}
\listoftables

\cleardoublepage

%%% OSREDNJI DEL

\pagestyle{fancy}
\fancyhead[CE,RE]{}
\fancyhead[LO,CO]{}
\fancyhead[LE]{\textbf{\nouppercase{\leftmark}}}
\fancyhead[RO]{\textbf{\nouppercase{\rightmark}}}

%\include{Uvod}

\chapter{Uvod}
\label{ch1}

\index{BibTeX}

N
%\include{Matemati"cni izrazi}

\chapter{Matematični izrazi}
\label{ch2}



\section{Osnovne enačbe gibanja}
\subsection{Newtonovi zakoni}





%\include{Slike in tabele}

\chapter{Slike in tabele}
\label{ch3}

% Slike in dalj"se tabele praviloma vklju"cujemo v dokument kot 
% plavajo"ce objekte ali plovke (angle"sko floats).
% Polo"zaj plovke v kon"cnem izdelku je odvisen od poteka besedila.
% "Ce "zelimo dolo"citi to"cno mesto plovke, ukazu \verb|\begin{figure}|
% ali \verb|\begin{table}| dodamo [dolo"cilo]:

\begin{itemize}
\item[---]{{\tt h} \hspace{1 cm} tukaj}
\item[---]{{\tt t} \hspace{1 cm} na vrhu strani}
\item[---]{{\tt b} \hspace{1 cm} na dnu strani}
\item[---]{{\tt p} \hspace{1 cm} na posebni strani}
\end{itemize}

\noindent
Slike in tabele potrebujejo podnapise s pojasnili. Vkolikor je slika povzeta iz drugega vira, mora biti tudi ta naveden:

% \begin{figure}[h]
% \begin{center}
% \includegraphics[width=10cm]{Bragglaw}
% \end{center}
% \caption[Braggov uklon.]{Braggov uklon je uklon oziroma sipanje 
% rentgenskih "zarkov na kristalni mre"zi. Pri tem pride v dolo"cenih 
% smereh zaradi interference do mo"cnih oja"canj. 
% Slika je povzeta iz \cite{Bragg}.}
% \label{pic1}
% \end{figure}

\index{plovke}



\newpage
\section{Formati slik}

% V \LaTeX{}ov dokument lahko vklju"cimo slike razli"cnih formatov. 
% Vedeti pa moramo, da program {\tt pdflatex} podpira ve"c formatov 
% kot {\tt latex}. Pri uporabi programa {\tt latex} lahko vstavljamo 
% slike edino v formatu PostScript (.ps ali .eps --- kon"cnica ni
% va"zna, le slika mora imeti definiran okvir, ki je zapisan
% v njeni datoteki, obi"cajno v formatu \%\%{\tt BoundingBox x1 y1 x2 y2}). 
% "Ce uporabljamo {\tt pdflatex}, so primerni formati na primer 
% .png, .pdf in .jpg.  Tudi slike v formatu .eps je mo"zno vstaviti,
% "ce tako kot v tem vzorcu uporabimo paket {\tt epstopdf}, ki vsako
% .eps sliko samodejno pretvori v obliko .pdf.  (Lahko pa seveda
% vsako .ps in .eps sliko "ze prej sami pretvorimo v sliko formata .pdf
% z istim programom in uporabljamo le .pdf slike.  To je morda
% celo najbolj priporo"cljiva pot.)  Strnjeno v Tabeli~\ref{tbl1}.




\chapter{Zaključek}

P

%%% LITERATURA

\cleardoublepage\phantomsection
\addcontentsline{toc}{chapter}{Literatura}
\bibliographystyle{myapsrevSLO}
\bibliography{Bibliografija}

%%% DODATKI

\cleardoublepage
\renewcommand\appendixname{Dodatek}
\begin{appendices}

\chapter{Naslov prvega dodatka}
    

\chapter{Naslov drugega dodatka}

\end{appendices}

%%% KAZALO (NEOBVEZNO)

\cleardoublepage
\printindex

\end{document}